%%%%%%%%%%%%%%%%%%%%%%%%%%%%%%%%%%%%%%%%%
% Short Sectioned Assignment
% LaTeX Template
% Version 1.0 (5/5/12)
%
% This template has been downloaded from:
% http://www.LaTeXTemplates.com
%
% Original author:
% Frits Wenneker (http://www.howtotex.com)
%
% License:
% CC BY-NC-SA 3.0 (http://creativecommons.org/licenses/by-nc-sa/3.0/)
%
%%%%%%%%%%%%%%%%%%%%%%%%%%%%%%%%%%%%%%%%%

%----------------------------------------------------------------------------------------
%	PACKAGES AND OTHER DOCUMENT CONFIGURATIONS
%----------------------------------------------------------------------------------------

\documentclass[paper=a4, fontsize=11pt, abstract=on]{scrartcl} % A4 paper and 11pt font size

\usepackage[top=0.2in, bottom=0.7in, left=0.7in, right=0.7in]{geometry}


\usepackage[T1]{fontenc} % Use 8-bit encoding that has 256 glyphs
\usepackage[utf8]{inputenc}
\usepackage{fourier} % Use the Adobe Utopia font for the document - comment this line to return to the LaTeX default
\usepackage[english]{babel} % English language/hyphenation
\usepackage{amsmath,amsfonts,amsthm} % Math packages

\usepackage{sectsty} % Allows customizing section commands
\allsectionsfont{\normalfont\scshape} % Make all sections centered, the default font and small caps

\usepackage{fancyhdr} % Custom headers and footers
\pagestyle{fancyplain} % Makes all pages in the document conform to the custom headers and footers
\fancyhead{} % No page header - if you want one, create it in the same way as the footers below
\fancyfoot[L]{} % Empty left footer
\fancyfoot[C]{} % Empty center footer
\fancyfoot[R]{\thepage} % Page numbering for right footer
\renewcommand{\headrulewidth}{0pt} % Remove header underlines
\renewcommand{\footrulewidth}{0pt} % Remove footer underlines
\setlength{\headheight}{13.6pt} % Customize the height of the header

\numberwithin{equation}{section} % Number equations within sections (i.e. 1.1, 1.2, 2.1, 2.2 instead of 1, 2, 3, 4)
\numberwithin{figure}{section} % Number figures within sections (i.e. 1.1, 1.2, 2.1, 2.2 instead of 1, 2, 3, 4)
\numberwithin{table}{section} % Number tables within sections (i.e. 1.1, 1.2, 2.1, 2.2 instead of 1, 2, 3, 4)

%\setlength\parindent{0pt} % Removes all indentation from paragraphs - comment this line for an assignment with lots of text

\usepackage[explicit]{titlesec}
%\titleformat{\section}{\normalfont\Large\bfseries}{}{0em}{\thesection \quad #1}[]
%\titleformat{name=\section,numberless}{\normalfont\Large\bfseries}{}{0em}{#1}[]
\usepackage{enumitem}

\newcommand\blfootnote[1]{%
    \begingroup
    \renewcommand\thefootnote{}\footnote{#1}%
    \addtocounter{footnote}{-1}%
    \endgroup
}

% User added packages
\usepackage{hyperref}
\usepackage{csquotes}
\usepackage{abstract}
\usepackage[backend=biber, style=numeric, natbib=true, sorting=nyt]{biblatex}
\bibliography{PA193-review-part01.bib}
\usepackage{csquotes}
\usepackage[page,toc]{appendix}
\usepackage{placeins}

\usepackage{graphicx}
\graphicspath{ {./pics/} }
\DeclareGraphicsExtensions{.pdf,.png,.jpg}

%----------------------------------------------------------------------------------------
%	TITLE SECTION
%----------------------------------------------------------------------------------------

\newcommand{\horrule}[1]{\rule{\linewidth}{#1}} % Create horizontal rule command with 1 argument of height

\title{
\normalfont \normalsize
%\textsc{PA193 Secure coding principles and practices} \\ [12pt]
\LARGE Docker Image Specification Parser \\
}

\author{\normalsize Milan Bartoš, Peter Kotvan, Matěj Plch} % Your name

\date{\normalsize\today} % Today's date or a custom date

\begin{document}

\maketitle % Print the title

For our project we chose to implement a prser of Docker Image Specification
using JSON data-interchange format. Docker is an open-source project aiming to
automate the deployment of applications inside software containers. It provides
additional layer of abstraction and automation of operating system level
virtualization on Linux. It is based on resource isolation features like cgroups
and kernel namespaces. Docker images are composed of layers. These layers refer
to metadata stored a json structures describing basic information about the
image as such as date created, author and the ID of its parent image as well as
execution/runtime configuration like its entry point, CPU/memory shares,
networking and volumes.

The parser consists of three independent modules providing lexical, syntactic
and semantic analysis of the input. First two modules together ensure the
correctness of the JSON format while the latter verifies the Docker Image
Specification.

\section{Lexical analysis}

The token class is basic data structure used to store information token type and
contents. In the tokenizer class there are implemented methods for processing
different types of tokens for example numbers, strings or braces. The get\_token
method reads the input characterwise and applies these methods to extract
tokens. With each call next token is returned.

\section{Syntactic analysis}


\section{Semantic analysis}

Semantic analysis is implemented in analysis.cpp file. It reads the tree constructed by syntax analysis. Analysis implements set of is\_* functions that check whether given json\_value is of expected type for that part of tree, or std::string has expected content for that particular element. By using these functions and tree search we are able to validate the tree from the semantic point of view. Required elements in Docker format are expected to exist and extra elements are ignored as they are seen as "implementation specific" by Docker file format specification.

\section{Issues}

During the work on the project our team encountered only two issues that we are
aware of.

\begin{itemize}

    \item After closer look at Docker Image Specification example we have
        realised that it was not a valid JSON since comma appeared after
        the last token in several objects. We decided to give docker a
        precedence so our code accepts such input as valid.

    \item During the phase of designing the parser we have forgotten to
        incorporate UTF-8 validation. We have realised this too late in the
        programming process and the decision was to read the input one more time
        with sole purpose of UTF-8 validation. This can be considered as a
        performance issue.

\end{itemize}

\section{Stats}

\textbf{TODO: update stats with final commit!}

\begin{verbatim}
-------------------------------------------------------------------------------
Language                     files          blank        comment           code
-------------------------------------------------------------------------------
JSON                            45             27              0           1533
C++                              9            195            108           1236
C/C++ Header                    16            130             84            384
XML                              1              0              3             83
Bourne Shell                     1              8              0             45
make                             4             18              0             45
YAML                             1              0              0             12
-------------------------------------------------------------------------------
SUM:                            77            378            195           3338
-------------------------------------------------------------------------------
\end{verbatim}

\end{document}

