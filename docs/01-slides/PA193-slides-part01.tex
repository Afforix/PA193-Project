% vim: set filetype=tex:ai:et:sw=4:ts=4:sts=4:tw=80
%%%%%%%%%%%%%%%%%%%%%%%%%%%%%%%%%%%%%%%%%
% Beamer Presentation
% LaTeX Template
% Version 1.0 (10/11/12)
%
% This template has been downloaded from:
% http://www.LaTeXTemplates.com
%
% License:
% CC BY-NC-SA 3.0 (http://creativecommons.org/licenses/by-nc-sa/3.0/)
%
%%%%%%%%%%%%%%%%%%%%%%%%%%%%%%%%%%%%%%%%%

%----------------------------------------------------------------------------------------
%	PACKAGES AND THEMES
%----------------------------------------------------------------------------------------

\documentclass{beamer}

\mode<presentation> {

% The Beamer class comes with a number of default slide themes
% which change the colors and layouts of slides. Below this is a list
% of all the themes, uncomment each in turn to see what they look like.

% \usetheme{default}
%\usetheme{AnnArbor}
% \usetheme{Antibes}
%\usetheme{Bergen}
%\usetheme{Berkeley}
%\usetheme{Berlin}
%\usetheme{Boadilla}
%\usetheme{CambridgeUS}
%\usetheme{Copenhagen}
%\usetheme{Darmstadt}
%\usetheme{Dresden}
%\usetheme{Frankfurt}
%\usetheme{Goettingen}
%\usetheme{Hannover}
%\usetheme{Ilmenau}
%\usetheme{JuanLesPins}
%\usetheme{Luebeck}
% \usetheme{Madrid}
%\usetheme{Malmoe}
%\usetheme{Marburg}
%\usetheme{Montpellier}
%\usetheme{PaloAlto}
%\usetheme{Pittsburgh}
%\usetheme{Rochester}
%\usetheme{Singapore}
%\usetheme{Szeged}
\usetheme{Warsaw}

% As well as themes, the Beamer class has a number of color themes
% for any slide theme. Uncomment each of these in turn to see how it
% changes the colors of your current slide theme.

%\usecolortheme{albatross}
%\usecolortheme{beaver}
%\usecolortheme{beetle}
%\usecolortheme{crane}
%\usecolortheme{dolphin}
%\usecolortheme{dove}
%\usecolortheme{fly}
%\usecolortheme{lily}
%\usecolortheme{orchid}
%\usecolortheme{rose}
%\usecolortheme{seagull}
%\usecolortheme{seahorse}
%\usecolortheme{whale}
%\usecolortheme{wolverine}

%\setbeamertemplate{footline} % To remove the footer line in all slides uncomment this line
%\setbeamertemplate{footline}[page number] % To replace the footer line in all slides with a simple slide count uncomment this line

%\setbeamertemplate{navigation symbols}{} % To remove the navigation symbols from the bottom of all slides uncomment this line
}

\setbeamertemplate{headline}{} % remove header

\usepackage{graphicx} % Allows including images
\graphicspath{ {./pic/} }
\usepackage{booktabs} % Allows the use of \toprule, \midrule and \bottomrule in tables
\usepackage[utf8]{inputenc}
\usepackage{url}
\usepackage{fancyvrb}
% \usepackage[backend=biber, style=numeric, sorting=nyt]{biblatex}
% \bibliography{PA193-slides-part01.bib}

%----------------------------------------------------------------------------------------
%	TITLE PAGE
%----------------------------------------------------------------------------------------

\title[Docker Img Spec Parser]{PA193: Docker image specification parser.} % The short title appears at the bottom of every slide, the full title is only on the title page

\author{\textbf{Milan Bartoš, Peter Kotvan, Matěj Plch}}
\institute[FI MUNI] % Your institution as it will appear on the bottom of every slide, may be shorthand to save space
{
FI MUNI \\ % Your institution for the title page
\medskip
}

%\date{15. 5. 2014} % Date, can be changed to a custom date

\begin{document}

\begin{frame}
\titlepage % Print the title page as the first slide
\end{frame}

%----------------------------------------------------------------------------------------
%	PRESENTATION SLIDES
%----------------------------------------------------------------------------------------

\section{Introduction}

\begin{frame}[fragile]{The Project}

\begin{itemize}
    \item Docker - software containers
    \item Isolation - cgroups and kernel namespaces
    \item Docker Image - template for distributing containers
    \item Docker Image Specification - JSON format
    \item C++
    \item Leak free
\end{itemize}

\end{frame}

\begin{frame}[fragile]{Spec example}
\begin{verbatim}
{
    "author": "foobar",
    "architecture": "amd64",
    "os": "linux",
    "Size": 271828,
    "config": {
        "User": "foobar",
        "Memory": 2048,
        "MemorySwap": 4096,
        "CpuShares": 8,
        "ExposedPorts": {
            "8080/tcp": {}

...
\end{verbatim}
\end{frame}

\begin{frame}{Structure}

\begin{itemize}
    \item Lexical analysis
    \item Syntactic analysis
    \item Semantic analysis
\end{itemize}
\end{frame}

\begin{frame}[fragile]{Lexical Analysis}

\begin{itemize}
    \item Read the input file characterwise
    \item Identify tokens
    \item Recognize token types
    \item Verify format of numbers and strings
\end{itemize}

\begin{Verbatim}[fontsize=\small]
enum token_type {
    T_EOF,      // End Of File
    T_ERR,      // Error
    T_LBRACE,   // { left brace
    T_COLON,    // : colon
    T_STR,      // string enclosed in double quotes
    T_NUM,      // number
    T_TRUE,     // true

...
\end{Verbatim}
\end{frame}

\begin{frame}{Syntactic Analysis}
\begin{itemize}
    \item class json\_tree\_builder
    \item performs top-down parsing
    \item accepts tokens and creates JSON nodes (shared pointers)
    \item nodes allocated using std::make\_shared - operator new never used
    \item must accept invalid JSON \{ "foo":"bar" , \}
\end{itemize}
\end{frame}

\begin{frame}{Semantic Analysis}
\begin{itemize}
	\item tree search and is\_* functions
	\item extra items are ignored - implementation specific
	\item some items content is checked, some not becuase of missing spec or just because they can be anything
\end{itemize}
\end{frame}

\begin{frame}{Technical realisation}
\begin{itemize}
    \item C++11, no dependencies
    \item never used operator new or malloc
    \item no manually written dtors
    \item Travis CI - build + tests
    \item Cppcheck, Flawfinder, PREfast, Coverity
\end{itemize}
\end{frame}

\begin{frame}{Problems and Limitations}

    \begin{itemize}

        \item Invalid JSON in example

        \item UTF-8 validation does duplicate file reading

    \end{itemize}

\end{frame}

\begin{frame}[fragile]{Statistics}

\begin{itemize}

    \item GitHub: 157 commits

\end{itemize}

\begin{Verbatim}
Language                     files           code
-------------------------------------------------
JSON                            54           1913
C++                              8           1185
C/C++ Header                    16            383
XML                              1             83
Bourne Shell                     1             54
make                             4             45
YAML                             1             12
-------------------------------------------------
SUM:                            85           3675
\end{Verbatim}

\end{frame}

\begin{frame}{Thank You}
    \Huge{\centerline{Thank you for your attention.}}
\end{frame}

%----------------------------------------------------------------------------------------

\end{document}
